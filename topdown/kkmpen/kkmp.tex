\documentclass{article}
\usepackage{amsfonts}
\usepackage{amsthm}
\usepackage{graphicx}

\oddsidemargin 0in
\evensidemargin 0in
\textwidth 6.5in
\topmargin 0in
\marginparsep .1in
\textheight 9in
\headsep 0in

\author{Chand T. John}

\title{$k$-Means Segmentation with Curvature Penalty}

\newtheorem{theorem}{Theorem}
\newtheorem{definition}{Definition}
\newtheorem{proposition}{Proposition}
\newtheorem{lemma}{Lemma}

\begin{document}
\maketitle
%\large{

% ABSTRACT %%%%%%%%%%%%%%%%%%%%%%%%%%%%%%%%%%%%%%%%%%%%%%%%%%%%%%%%%%%%%%%%%%%

\begin{abstract}
Here we combine Lloyd's $k$-means segmentation algorithm with a segmentation
algorithm based on geometric distances along with a penalty against including
high-curvature edges in segments in order to segment human teeth represented
as closed triangle meshes.
\end{abstract}

% LLOYD'S ALGORITHM %%%%%%%%%%%%%%%%%%%%%%%%%%%%%%%%%%%%%%%%%%%%%%%%%%%%%%%%%%

\section{Lloyd's Algorithm}
It is common to want to cluster a given set of points into $k$ different
groups.  Any algorithm that solves this problem is a $k$-means clustering
algorithm.  A popular $k$-means clustering algorithm is {\em Lloyd's
Algorithm} \cite{kmeans}.  In our case, we are clustering faces on a closed
triangle mesh into $k$ segments rather than clustering points into $k$
clusters.

Lloyd's Algorithm consists of two stages.  First we must pick an initial
subdivision of the mesh into $k$ segments, where $k$ is an integer (usually
greater than 2) picked by the user.  For teeth a typical value is $k=10$.  A
common initial segmentation method is as follows.  Pick $k$ random faces
$f_1, \ldots, f_k$ on the mesh.  These faces represent centers of the clusters
$C_1, \ldots, C_k$, respectively.  Now for each remaining face $f$ on the
mesh, we find which $f_i$ is closest to $f$ and add $f$ to the cluster $C_i$.
In the second stage of Lloyd's Algorithm, which we call {\em expectation
maximization} or EM (due to its similarity to the actual EM algorithm), we
repeat the following two steps an arbitrary (i.e. user-defined) number of
times:
\begin{enumerate}
\item Recompute the centers $c_1, \ldots, c_k$ of clusters $C_1, \ldots, C_k$.
\item Recluster the faces of the mesh using the newly computed centers to get
a new set of clusters $C_1', \ldots, C_k'$.
\end{enumerate}

% GEOMETRIC SEGMENTATION ALGORITHM %%%%%%%%%%%%%%%%%%%%%%%%%%%%%%%%%%%%%%%%%%%

\section{Geometric Segmentation Algorithm}
Katz and Tal \cite{katztal} describe a segmentation algorithm for triangle
meshes that tries to segment out protruding parts of a mesh by incrementally
choosing $k$ segment centers which are maximally distant from each other (so
the representative face of a protruding segment will likely be at the
outermost tip of the segment).  Their paper actually contributes a more
complex algorithm than the version of the algorithm we will use here:
\begin{enumerate}
\item Compute which face's centroid has the greatest sum of distances from all
other centroids (i.e., find the face which is most protruding of all faces in
the mesh); label this face as the first representative face $f_1$.
\item For $i = 2, \ldots, k$: choose $f_i$ to be the face with maximum sum of
distances from all other representative faces $f_1, \ldots, f_{i-1}$, where
distance between two faces is defined as the distance between their centroids.
\item The faces $f_1, \ldots, f_k$ are now representative faces for their
segments $C_1, \ldots, C_k$, respectively.  Initially, $C_1 = \{f_1\}$,
$C_2 = \{f_2\}$, and so on.  Now for each face $f$ that has not yet been
assigned to a segment, compute the distances from $f$ to each
representative face and choose the face $f_i$ closest to $f$.  Add $f$ to the
corresponding segment $C_i$.
\end{enumerate}

% COMBINED GEOMETRIC-LLOYD ALGORITHM %%%%%%%%%%%%%%%%%%%%%%%%%%%%%%%%%%%%%%%%%

\section{Combined Geometric-Lloyd Algorithm}
In either of the two algorithms above, we can use either geometric distances
or geodesic distances.  Geodesic distances are substantially slower to
compute but are topologically more relevant.  This can be experimented with
in implementation.

We can now combine the algorithms so that the Katz-Tal geometric segmentation
algorithm is used as the initial segmentation and then the expectation
maximization step of Lloyd's Algorithm is applied to refine the segmentation
over a number of iterations.

% THE GEODESIC HIGH CURVATURE PENALTY %%%%%%%%%%%%%%%%%%%%%%%%%%%%%%%%%%%%%%%%

\section{The Geodesic High Curvature Penalty}
The combined Geometric-Lloyd Algorithm consists of two main stages: the
geometric segmentation stage and the Lloyd algorithm stage.  It is not clear
how to add a penalty to the initial geometric segmentation stage to prevent
high-curvature edges from being put into the interior of a segment.  However,
there is a fairly natural and easily implementable way to add such a penalty
to the Lloyd algorithm stage.  We now describe how this penalty is measured
and added to the combined Geometric-Lloyd Algorithm.

The Lloyd algorithm stage consists of simply taking an existing segmentation,
recomputing the centers of the segments, and resegmenting the shape by putting
each face in the segment whose center is the closest to the face.  We choose
to modify this resegmentation stage by instead choosing the segment in which
to place a particular face by the following method.  For a given face $f$,
let $c_1$ and $c_2$ be the two centers closest to $f$.  Let $f_1$ and $f_2$ be
the faces on the mesh closest to $c_1$ and $c_2$, respectively.  Now consider
the geodesic path $p_1$, i.e. the path along the dual mesh, from face $f$ to
face $f_1$.  Also consider the path $p_2$ from face $f$ to face $f_2$.  We
measure how much curvature is on the paths $p_1$ and $p_2$ to decide whether
$f$ belongs to the segment corresponding to $c_1$ or the segment corresponding
to $c_2$.  We wish to put $f$ in a segment where the path from $f$ to the
segment center contains lower curvature edges.  To measure how much curvature
there is on any such path $p$ through the dual mesh, we use an
{\em unbelongingness function}
\[ U(p) = \sum_{\mbox{$f,f'$ consecutive faces in $p$}} (1-\kappa(f,f')) \]
where $\kappa(f,f')$ is the dot product of the oriented normal vectors of
faces $f$ and $f'$, which is closer to $-1$ when the curvature of the edge
between $f$ and $f'$ is higher, and closer to $1$ when the curvature is lower.
We could experiment with other unbelongingness functions, such as $U(p)=
\max_{f,f'} \kappa(f,f')$.  We shall stick with our initial unbelongingness
function for now.  Now when we are deciding what segment to put $f$ into, we
simply compute $U(p_1)$ and $U(p_2)$ and put $f$ into the segment whose
geodesic path from its center to $f$ has the lower unbelongingness function
value.  That is, if $U(p_1) < U(p_2)$, we add $f$ to the segment whose center
is $c_1$, and otherwise we add $f$ to the segment whose center is $c_2$.

% GEOMETRIC-LLOYD ALGORITHM WITH PENALTY %%%%%%%%%%%%%%%%%%%%%%%%%%%%%%%%%%%%%

\section{Geometric-Lloyd Algorithm with Penalty}
The final algorithm combining the Geometric-Lloyd Algorithm with the
geodesic high-curvature penalty works as follows.  First, the Katz-Tal
segmentation algorithm is applied to a closed mesh to obtain an initial
segmentation of the mesh.  Then the expectation maximization stage of Lloyd's
Algorithm is applied an arbitrary number of times.  But each time we apply
a step of expectation maximization, instead of directly putting each face
into the segment whose center face is nearest to it, we measure the
unbelongingness of the face to its {\em two} nearest segment centers based on
its geodesic path to those two centers, and place the face into the segment
to whose center the face has lower unbelongingness (i.e. the face is placed in
the segment to whose center the face's geodesic path has lower curvature by
whatever criterion is used by the unbelongingness function).

% BIBLIOGRAPHY %%%%%%%%%%%%%%%%%%%%%%%%%%%%%%%%%%%%%%%%%%%%%%%%%%%%%%%%%%%%%%%

\begin{thebibliography}{99}

\bibitem{kmeans} T. Kanungo, D. M. Mount, N. S. Netanyahu, C. D. Piatko, R.
Silverman, A. Y. Wu.  An Efficient $k$-Means Clustering Algorithm: Analysis
and Implementation.  {\em IEEE Transactions on Pattern Analysis and Machine
Intelligence}, pp. 881-892, 2002.

\bibitem{katztal} S. Katz, A. Tal.  Hierarchical Mesh Decomposition using
Fuzzy Clustering and Cuts.  {\em ACM Transactions on Graphics}, Vol. 22 (3),
pp. 954-961, 2003.

\end{thebibliography}

%} % end of large
\end{document}



